\iffalse
\begin{enumerate}
    \item m-MIMO FDD OFDM MATLAB Simulation and code
    \begin{itemize}
        \item Simulation able to simulate a Base Station (BS) with a m-MIMO antenna system one or more UE connected to it
        \item Generate and save the necessary CSI for the downlink channel
        \item Generate labelled CSI and YOLO image data for deep learning training
        \item Change between different downlink channel reconstruction techniques
    \end{itemize}
    \item Introduce a new deep learning technique for extracting the paths from the YOLO channel images
    \item Investigate alternative approaches for extracting bright spots in images using simple coloured image analysis techniques
    \item Production of a comprehensive paper comparing the different channel reconstruction techniques using several comparative metrics and including necessary background and discussion of results
\end{enumerate}
\fi

The proposed project will produce several deliverables which will contribute to the research and techniques of m-MIMO downlink channel reconstruction. A  configurable MATLAB m-MIMO simulator will be created and made available on Github to enable future research and channel reconstruction technique analysis. Building off of the work presented in \cite{Li2020}, the YOLO CSI and images generated by the project's MATLAB simulator could be used to train other machine learning models to improve or enhance the current deep learning channel reconstruction techniques for m-MIMO systems. As discussed in the sections above, the authors plan to make their own modifications to the YOLO deep learning algorithm to estimate the channels with higher precision and investigate other simple image analysis techniques to reduce the computation time of channel estimation. Finally, a comprehensive paper detailing and comparing the modern downlink channel reconstruction techniques will be presented in a final report to discuss our findings. Several comparative metrics will be used in the assessment of the different techniques outputting the error and accuracy of each across different conditions to provide evidence for selecting a particular technique.