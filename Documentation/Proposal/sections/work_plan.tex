\iffalse
\begin{enumerate}
    \item 5G m-MIMO FDD Simulation with 1 to n UE
    \begin{enumerate}
        \item Begin simulation with one UE using m-MIMO with multiple paths
        \item MATLAB m-MIMO simulation using available MATLAB 5G and communication toolbox code as a base example to build custom channels and desired m-MIMO system architecture
        \item Generate 4000 or more labelled CSI and image data for YOLO and other deep learning algorithms
        \item Implementation of channel reconstruction techniques in MATLAB
        \item Vary the parameters of the m-MIMO system, such as the number of antennas, number of subcarriers, number of propagation paths, Signal to Noise Ratio (SNR) to test the effectiveness of the different channel reconstruction algorithms with different m-MIMO configurations and to simulate different environments
        \item Introduce more than one UE to analyze the effect of the paths and noise of the channel
    \end{enumerate}
    \item Generate necessary Channel State Information (CSI) to be able to determine the dominate channels of a Uniform Linear Array (ULA) antenna array
    \item Reconstruct the downlink channel using the methods including the new innovative deep learning technique YOLO, along with NOMP, MUSIC, and maybe LASSO for comparison between different channel reconstruction techniques
    \item Investigate simple image analysis techniques to extract the bright spots from YOLO's channel images
    \item For deep learning algorithms, use the generated labelled channel image data from the YOLO technique to train the learning models to extract the delay and angle estimation values
    \item To avoid common pitfalls of machine learning, such as over or under fitting, Cross Validation, Regularization, and performance metrics will be used to evaluate and compare YOLO and other machine learning algorithms being used in the experiments
    \item Compare the observed channel of each channel reconstruction method to the actual channel, exact details derived from MATLAB simulation, using statistical error functions, including Mean Squared Error (MSE) as presented in \cite{Li2020}
    \item Compare the channel reconstruction methods running time to reconstruct the downlink channel paths
    \item Compare the spectrum efficiency of each technique
    \item During the comparison and analysis, the variation of the m-MIMO parameters should be considered to determine which techniques function better in their respective environements and configurations
\end{enumerate}
\fi
To accomplish our project objectives, the project is split into several phases, each generating a different deliverable and data for channel reconstruction technique comparison. To begin, a FDD m-MIMO system will be developed in MATLAB to enable the simulation of downlink and uplink channels between one, or several, UE and the m-MIMO Base Station (BS). Next, downlink channel reconstruction techniques, such as NOMP, MUSIC, and YOLO, will be implemented alongside the MATLAB simulation to extract the necessary data needed from channel reconstruction. After, the authors will introduce modifications to YOLO's deep learning algorithm with the aim to improve the accuracy and spectrum efficiency of the algorithm. Finally, the channel estimation techniques for downlink channel reconstruction will be compared against several metrics.

The first step of the project will be to produce a simulator in MATLAB of a m-MIMO system with varying parameters. The m-MIMO simulation will be able to simulate a m-MIMO with a different number of antennas, sub-carriers, propagation paths to test the downlink channel reconstruction techniques in a variety of m-MIMO configurations. The Signal to Noise Ratio (SNR) will also be dynamic and randomly generated in the simulation to simulate environments with different noise levels and test the chosen techniques in varying environmental conditions. The m-MIMO will be connected to one simulated UE to generate and simulate the data exchanged between the UE and the m-MIMO BS across the downlink and uplink channels. If time permits it, then more than one UE will be introduced to study the effect of multiple users communicating with the m-MIMO BS with regards to the ability of the channel reconstruction techniques to extract the downlink channels while multiple users are sharing the spectrum. The simulation will produce the CSI, for exact channel details, and uplink pilots received at the m-MIMO BS. The CSI will be used to construct the actual channel details for later comparison and the details of the pilots will be used by the channel reconstruction techniques to extract the estimated channel.

The channel reconstruction techniques defined in \cite{Li2020} will be implemented in MATLAB, or another programming language, to estimate the downlink channel from the m-MIMO to the UE and select the optimal propagation paths. The deeping learning technique, YOLO, defined in \cite{Li2020} will require a large amount of labelled data to train its machine learning model and is the main algorithm of interest. The CSI and uplink information generated in the simulator can be used to produce the labelled channel images for training of the YOLO algorithm. The training parameters used by the authors of \cite{Li2020} are defined in the text. Similar parameters will be used when training our deep learning algorithm, but to avoid common pitfalls of training, such as over or under fitting, we will also introduce Cross Validation, Regularization, and several performance metrics to monitor the training of the deep learning algorithm.

Following the implementation of the channel reconstruction techniques, the authors will investigate and introduce a few novel approaches and modifications to the original YOLO algorithm to improve its performance, accuracy, and efficiency. Different machine learning techniques can be explored and reviewed to determine whether using another algorithm would be beneficial. Other image analysis techniques that extract the bright propagation paths of the YOLO images without deep learning will also be investigated to avoid training and produce a more simple algorithm thereby reducing computation time.

Once the channel reconstruction techniques have been implemented and trained, they shall be deployed against the varying parameters and noise coniditions of the m-MIMO simulator for a direct comparison. The comparison of the channels will include statistical error functions, such as the Mean Squared Error (MSE), comparing the exact downlink channels to the estimated channels predicted by the reconstruction techniques. The techniques running times will also be compared along with the spectrum efficiency of each technique in a variety of conditions.