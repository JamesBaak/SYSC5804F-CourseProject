Telecommunication systems demand constant innovation, the requirements grow exponentially with each generation of wireless technology. One of the ways 5$^{th}$ Generation (5G) wireless systems respond to these growing requirements is through the use of massive Multi-Input Multi-Output (m-MIMO) antennas \cite{Dahlman2018}. These m-MIMO systems have the potential to improve the channel bandwidth, coverage, and capacity through beamforming and spacial multiplexing. To take advantage of these perks the systems require accurate Channel State Information (CSI). However, the more antennas a system has, the larger the channel matrix, further complicating the channel sounding process. m-MIMO systems have increased amount of overhead when acquiring CSI, specifically in the down-link direction \cite{Mawatwal2020}. Channel reciprocity is a technique used to reduce the amount of feedback overhead in m-MIMO systems. Specific aspects of the channel are frequency independent, such as path delay and angle. These aspects can be measured in the uplink direction, instead of fed back from the UE. Additionally,  m-MIMO channels typically have many more antennas then dominant paths, causing them to be sparse in the delay-angle space. Classical algorithms to calculate the delay and angle include  Newtonized Orthogonal Matching Pursuit (NOMP) \cite{Han2019}, Multiple Signal Identification Classification (MUSIC) \cite{Yin2016} and Least Absolute Shrinkage and Selection Operator (LASSO) \cite{liu2016}. Researchers have also proposed using Machine Learning (ML) techniques to such as You Only Look Once (YOLO) to extract the path delays and angles from received pilots \cite{Li2020}. Improving channel estimation and reducing the overhead of CSI feedback is an active area of research for 5G systems that will be explored through this term project.

